\begin{table}[H]
  \centering
  \small
  \singlespacing
  \begin{tabular}{llrr}
  \toprule
   &  & \# Poems & \# per model (avg.) \\
  Prompt type & Prompt &  &  \\
  \midrule
  \multirow[t]{8}{*}{Rhymed} & Write a long poem that does rhyme. & 5 & 2 \\
   & Write a poem (with 20+ lines) that rhymes. & 4 & 1 \\
   & Write a poem in ballad stanzas. & 5 & 2 \\
   & Write a poem in heroic couplets. & 7 & 2 \\
   & Write a poem in the style of Emily Dickinson. & 8 & 3 \\
   & Write a poem that does rhyme. & 6 & 3 \\
   & Write a rhyming poem. & 7 & 2 \\
   & Write a rhymed poem in the style of Shakespeare's sonnets. & 13 & 4 \\
  \cline{1-4}
  \multirow[t]{8}{*}{Unrhymed} & Write a long poem that does NOT rhyme. & 7 & 2 \\
   & Write a poem (with 20+ lines) that does NOT rhyme. & 7 & 2 \\
   & Write a poem in blank verse. & 5 & 2 \\
   & Write a poem in free verse. & 10 & 2 \\
   & Write a poem in the style of Walt Whitman. & 7 & 2 \\
   & Write a poem that does NOT rhyme. & 57 & 28 \\
   & Write a short poem that does NOT rhyme. & 6 & 2 \\
   & Write an unrhymed poem. & 5 & 2 \\
  \cline{1-4}
  \multirow[t]{6}{*}{Rhyme unspecified} & Write a long poem. & 7 & 2 \\
   & Write a poem (with 20+ lines). & 7 & 2 \\
   & Write a poem in groups of two lines. & 2 & 1 \\
   & Write a poem in stanzas of 4 lines each. & 6 & 2 \\
   & Write a poem. & 4 & 1 \\
   & Write a short poem. & 5 & 2 \\
  \bottomrule
  \end{tabular}
  \caption{Number of poems generated for each prompt.}
  \label{tab:num_poems_rhyme_promptings}
\end{table}